%%%%%%%%%%%%%%%%%%%%%%%%%%%%%%%%%%%%%%%%%%%%%%%%%%%%%%%%%%%
% EPFL report package, main thesis file
% Goal: provide formatting for theses and project reports
% Author: Mathias Payer <mathias.payer@epfl.ch>
%
% This work may be distributed and/or modified under the
% conditions of the LaTeX Project Public License, either version 1.3
% of this license or (at your option) any later version.
% The latest version of this license is in
%   http://www.latex-project.org/lppl.txt
%
%%%%%%%%%%%%%%%%%%%%%%%%%%%%%%%%%%%%%%%%%%%%%%%%%%%%%%%%%%%
\documentclass[a4paper,11pt,oneside]{report}
% Options: MScThesis, BScThesis, MScProject, BScProject
\usepackage[MScThesis,lablogo]{EPFLreport}
\usepackage{amsmath,amssymb}
\usepackage[utf8]{inputenc}
\usepackage{csquotes}
%\usepackage{natbib}
\usepackage{xspace}

\title{Learning Interpretable Abstract Representations in Reinforcement Learning Environments via Model Sparsity}
\author{Sergei Volodin}
\supervisor{Dr. Johanni Brea}
\adviser{Prof. Wulfram Gerstner}
%\coadviser{Second Adviser}
\expert{Carl-Johann Simon-Gabriel}

\dedication{\begin{raggedleft}
        The strongest evidence that we can obtain for the validity of a proposed induction method, is that it yields results that are in accord with intuitive evaluations in many different kinds of situations in which we have strong intuitive ideas.\\
        --- Ray J. Solomonoff\\
    \end{raggedleft}
}

\acknowledgments{
I thank my parents for the support in my studies, Switzerland and EPFL in particular for making it possible.
}

\address{EPFL IC/SV Laboratory of Computational Neuroscience (LCN)\\
    Bâtiment AAB \\
    offices 135-141 \\
    CH-1015 Lausanne}

\begin{document}
    \maketitle
    \makededication
    \makeacks

\begin{abstract}
"I choose this restaurant because they have vegan sandwiches" could be a typical explanation we would expect from a human. However, current Reinforcement Learning (RL) techniques are not able to provide such explanations, when trained on raw pixels.
RL for state-of-the-art benchmark environments are based on neural networks, which lack interpretability, because of the very factor that makes them so versatile -- they have many parameters and intermediate representations.
The best result that could be obtained are two rollouts from the internal model of the agent, showing that one policy would result in a higher reward than the another policy. These rollouts would be high-dimensional objects that are hard to compare together or reason about.
Additionally, feature attribution could be applied to these rollouts, which would show that certain parts of the raw image observations influence the outcome more than others. However, this would require a careful manual analysis of the feature attribution maps for every possible input that the agent can encounter.
In contrast, humans use shorter explanations that capture only the essential parts. This is possible because the world contains hierarchical structures in which only a few factors are important, and the rest can be ignored.
Interpretability of the resulting agent is crucial if such an agent is deployed in the real world in mission-critical scenarios, because a black-box neural network could give wrong predictions on unexpected inputs, without even giving an ability to know the reason of the failure. Thus, interpreting Reinforcement Learning agents is an important step towards safely deploying them.
In addition, the quest to mimic human-like reasoning is of general scientific interest, as it sheds light on the easy problem of consciousness.

To make an agent that can reason like a human does, we first need to learn a representation that corresponds to high-level objects contained in the environment, and their properties, like the representation that the natural language uses. If we lack such a representation, any reasoning algorithm's outputs would be useless for interpretability, since even the "referents" of the "thoughts" of such a system would be obscure to us: it would take many words of natural language to explain one concept that the system uses. Thus, a subgoal of making interpretable agents is to create representations that allow for simple explanations of causes and their effects in a given environment.

One way to define simplicity of causal explanations is the sparsity of the Causal Model that describes the environment: the causal graph has the fewest edges connecting causes to their effects. For example, a model for choosing the restaurant that only depends on the cause "vegan" is simpler and more interpretable than a model that looks at each pixel of a photo of the menu of a restaurant.

In addition, sparse causal models describing the latent dynamics are hypothesized by Yoshua Bengio to be crucial in the functioning of human consciousness, since human explanations have only few important causes.

In this thesis, we propose a framework for model-based Reinforcement Learning where the model is regularized for simplicity in terms of sparsity of the causal graph it corresponds to.
The framework contains a learned mapping from observations to latent features, and a model predicting latent features at the next time-steps given ones from the current time-step. The latent features are regularized with the sparsity of the model, compared to a more traditional regularization on the features themselves.
We test this framework on benchmark environments with non-trivial high-dimensional dynamics and show that it can uncover the causal graph with the fewest edges in the latent space.

For the future work, the causal graph can be used for easier grounding on a set of natural language explanations, as the agent now "knows" about the high-level concepts that the environment contains. In addition, a simple model would likely be less prone to overfitting, which could result in an improved robustness of such agents to distributional shift.
\end{abstract}

\maketoc

%%%%%%%%%%%%%%%%%%%%%%
\chapter{Introduction}
%%%%%%%%%%%%%%%%%%%%%%

Reinforcement learning is a general paradigm in Artificial Intelligence (AI) that considers two entities: an environment and an agent which interact with each other over multiple time-steps. The agent takes actions in the environment, and the environment gives the agent an observation representing the state of the environment, and a reward. The goal of the agent is to execute actions that lead to highest total reward during the interaction.

While RL environments in general can have arbitrary complexity, practically interesting environments usually have a certain low-dimensional structure in them. For example, in the popular CartPole environment, the observation (as image) has 720 000 variables, while the dynamics of the cart can be explained using only 4 variables with a simple Newtonian update rule. Atari games, while having 100 800 variables in an observation, can be described using only 128 bytes (of the RAM state). The dynamics in the pixel space would be complex, with one pixel at the next time-step potentially depending on all of the pixels on the previous one. However, the dynamics in the latent space is much simpler. The real world has this property as well: while a camera can capture many megapixels per second observing a drone, the equations describing the dynamics of the drone only have a few variables, and are much simpler.


RL agents could (and sometimes already do) solve real-world problems, such as steering the wheel of an autonomous car, controlling a plant, or determining which content to recommend to people on a social network. However, in many applications there exist highly undesirable outcomes, such as the car crashing due to a mistake in controlling it, or showing polarizing or controversial content to people. To prevent such cases, we need to have an understanding of what makes an agent take a certain action, why it "thinks" it is optimal, and why it "thinks" it does not violate any sensitive constraints. Since the most capable agents are based on neural networks (NNs), it is hard to interpret them. Both the large number of parameters in the network, and many intermediate representations at each layer -- the distinctive features that allow NNs to be so versatile in fitting different kinds of data, lead to problems with interpretability (or explainability), because apriori there is no structure in the network: all variables depend on all variables of the input.

Even the state-of-the-art approaches that look at individual neurons and try to determine their functions (such as a neuron responding to the images of dogs) do not allow to obtain a concise explanation, since, while some neurons are more important than others in predicting a particular outcome, all neurons contribute a non-zero amount to the outcome. Because of this, it becomes impossible to determine in advance what the outcome of a network would be in all of the input scenarios.

In order to make an agent interpretable (be able to answer questions of the form "why an action was taken given a particular observation"), the agent needs to grasp the high-level concepts and objects contained in the environment. This is a Representation Learning task, with an objective of obtaining representations allowing for good explanations. In addition, regardless of the task the agent is solving, it has to be able to explain in simple terms the effects of its actions on the environment.

While deep neural networks allow to fit various kinds of data, but lack interpretability due to the large number of parameters, in contrast, traditional ("good-old-fashioned" or GOFA) Artifical Intelligence (AI) approaches are interpretable (explainable) but lack the versatility of neural networks: they are not applicable to all the tasks.
Such GOFA approaches are characterized by discrete symbols that are the basis of a reasoning system, i.e. first-order logic.
These symbols correspond to real-world concepts and objects, such as the coordinates of a robot on the floor.
The drawback of such systems is that the mapping from real-world "messy" data (e.g. images from a camera) to discrete symbols (e.g. the coordinates) is hand-designed and thus requires effort.
Unifying these two approaches yields the best of both worlds, with the versatility of deep AI in being able to fit almost any dataset and the explainability of traditional AI.
To do so, we learn the representation of observations using a deep network. These representations are used by a system involving discrete components to make explainable predictions about the environment.

The quality of explanations given by such an agent would depend on the learned representation. Specifically, a representation of the observation in which every variable depends on every variable would be practically useless: an action potentially changes all of the variables, and, therefore, an explanation of action's effect would require to list all of them. Since the length of an explanation is crucial, the effect of an action should be describable by a change in only a few variables.

The current explainable RL agents can be classified into three following categories. First, many projects learn sparse representations in terms of the feature vector sparsity: the predicted learned features associated with an observation are sparse at each time-step as a vector. Such a condition, however, is neither necessary nor sufficient for the {\em explanations} of the dynamics to be simple. Indeed, any sparse coding technique will yield sparse feature vectors (for example, by assigning the first $k$ most sparse vectors to all the $k$ states of the MDP in case of a finite MDP), but a feature at the next time-step would depend on many features at the previous time-step. Indeed, without any assumption on the way the sparse features are obtained, two neighboring states can have drastically different representations. Thus, an action potentially changes all of the variables. This demonstrates that sparse feature vectors are not sufficient to have simple explanations. On the other hand, it is possible to have simple explanations without sparse feature vectors. Indeed, the state of a CartPole environment has all of the components being non-0 most of the time, and, yet, these components comply with a simple explainable kinematics equation. Thus, sparse feature vectors are not necessary for simple explanations.

Secondly, feature attribution is applied to the convolutional networks representing policies, and the analysis of circuits of the network is performed. This approach gives a) the parts of the image most relevant to predict the action and b) the visualization of what each neuron in the network is activated with. While such an approach is extremely robust to the type of the underlying agent (the analysis is performed after the training and does not require any changes in the architecture), the drawback is that it should be performed for every input image: we do not know how a particular new image would affect the outputs, since each output action still depends on all of the pixels.

Finally, a line of work is investigating {\em causal} explanations of the agents' actions. This means that a causal graph corresponding to the environment is obtained, and an explanation for an action is then a simple path-finding from the action node to the reward node in this graph. The drawback of the current projects is that the graph is either given manually (by hand), or learned with significant manual work, such as designing the feature space. In this project, we extend this line of work by making the graph learnable end-to-end from the raw observations.

We would like to automatically uncover the simple structure in the underlying complex high-dimensional observations that the environment dynamics generates -- uncover the succinct "laws of physics" that the environment operates with.
This would make the actions of an agent explainable.
It makes the actions of the agent explainable in terms of the "laws of physics" that we learn. In terms of Model-Based RL (an RL agent equipped with a model of the environment), we would like to learn a simple model of the environment.
Specifically, we would like each variable in the model to depend on the fewest possible variables at the previous time-steps. This would make actions explainable in terms of the variables an action changes, and the explanation would be short. The model is the "good-old-fashioned" component, since it includes discrete elements: the variable's causes consist of a set of elements, rather than an array of numbers of weights for all possible causes.
Mathematically, we require sparsity of the model's graph of dependencies -- the predicted output for a variable depends only on a few variables at the previous time-step. If represented as a graph (with cause-variables at the current time-step as parents and effect-variables at the next time-step as children), this graph needs to have fewest possible edges.

To allow for such a simple {\em model}, we need to change the representation, from raw high-dimensional observations to the features, which the model predicts for the next time-step. Such features are obtained from a deep learning {\em decoder}\footnote{We call it a decoder rather than an encoder because the observations are already {\em encoded} by the function mapping the simple environment states such as the cart's position and velocity into the high-dimensional observation}, a function mapping observations to features. Given a decoder, the model can be obtained by simply fitting it in a supervised way, and only selecting the cause variables that lead to an improvement in the loss. Therefore, the decoder determines the simplicity of the model.

Thus, the decoder needs to be trained jointly with the model, and it is indirectly regularized by the model: the model is directly regularized for sparsity (in terms of the number of edges), and the decoder needs to output data that such model can predict.
In our approach, the decoder is a neural network, and a model is also a neural network with a special mechanism to enforce the sparsity of the architecture.

If the the decoder is not additionally regularized for non-degeneracy, it is possible for the system to always predict constant features. This correspond to closing eyes and "predicting" successfully that everything that follows will be dark. The decoder needs to preserve the information that is relevant to the task the system is required to solve. Both in the real world, and in games, not all parts of the observations need to be predicted to achieve goals. For example, the color of an obstacle that a robot encounters is likely not important when it comes to avoiding it. However, this is task-specific, because, for example, a table is much less dangerous as an obstacle compared to water (a robot will be damaged if put into water, while an encounter with a table will likely only result in a loss of time when accomplishing a goal). Therefore, the model needs to predict task-specific features.

In this project, we use the two popular types of regularization: we either predict the reward (thus, only the aspects of observations relevant for predicting the reward are kept), or the whole observation (all the aspects of the observation are represented in the latent features). This is achieved by introducing a {\em reconstructor} -- a network that predicts either the reward of the full observation given the latent features.

Our approach allows to obtain "laws of physics" of an environment, or, more specifically, a sparse causal model representing the environment's dynamics.
If an environment allows for a complete separation of some parts of its dynamics (for example, termination of the episode only depends on the health of the player, and another variable, such as the number of ammunition does not affect the health), out approach would capture it, because a graph with two components has less edges than a connected graph. This way, we {\em disentangle} the representation, as it consists of (recurrent) {\em independent} mechanisms.
Such a model is very useful for explanations, because an effect of each action can be described in terms of changes in only a few variables (out of many). For example, for Cartpole with images as observations, our approach is expected to output the 4 features representing cart's and pole's position and velocity.

The main challenge when implementing our approach is combining all the requirements (sparsity of the causal model, learning the decoder, and the non-degeneracy of the decoder) into a single training procedure. Specifically, we propose and compare several methods (simple aggregation of losses with coefficients, primal-dual method for optimization under constraints, and an adaptive scheme for choosing the loss coefficients) and employ several useful tricks (such as using a {\em relative} Mean Squared Error (MSE) loss to have an interpretable magnitude of the loss, and adding the sampled binary mask of selected cause features to the model that predicts the effects), and use different methods to enforce sparsity (from $l_1$-regularization to discrete variables with various methods to compute the gradients). We present an ablation study to show the effects of various tricks and techniques used.

We design a simple toy environment to illustrate learning of sparse causal models from high-dimensional data, and show that our approach successfully obtains the minimal causal graph on it. Additionally, we illustrate the same positive behavior on a grid-world environment. Surprisingly, the method find a simpler (but correct) graph for the toy environment than the authors anticipated.

{\bf Contribution.} This thesis presents, up to our knowledge, the first successful result on learning a sparse causal model jointly with the representation on a challenging high-dimensional problem. While the parts of the "recipie" used, such as discrete variables to learn a sparse causal graph can be found in existing literature, combining them into a working system that learns the representation and the sparse causal graph, along with the required modifications and the techniques, is a novel main result. We briefly mention the implications of our work on the Integrated Information Theory of consciousness.

The result of this work can be used to improve significantly the explainability of Reinforcement Learning agents. First, a learned causal graph opens the possibility to drastically reduce the amount of data required to {\em ground} the various aspects of the observations with natural-language sentences.
In this way, a dataset containing only two sentences "cart going left" and "cart going right" with two data points will likely result in a mapping of the sentences to the learned velocity variable. In contrast, if we ground raw images this way, the model is likely to overfit to the particular details that the image has (spurious correlations), and will not depend on the true cause.
Sparse causal graph-based or natural language-based explanations of agent's actions would allow for faster certification of agents for mission-critical applications.
Secondly, since the learned model the simplest one, it is likely to be less prone to overfitting. In this way, it would be easier to adjust the agent to a novel environment either without modifications (because the model relies more on a discrete set of concepts, than on the whole vector of features), or with re-training the decoder only (for example, if the shape of an obstacle has changed, only the decoder needs to be updated). For the latter part, the training should take less time, because the bulk of the problem is already solved -- the model knows how to avoid obstacles.
Third, out prior to have the simplest model could make the agent require less data to train: since it expects the underlying dynamics to be simple, it will converge to the true model faster, because a more complex model (that a regular agent is likely to take as a first guess) will be discarded.
Next, having a disentangled representation with independent mechanisms allows to reuse components in the model, and learn it even faster. Many environments share the same dynamics -- for example, navigating in a 2-dimensional maze with 4 actions (up, down, left, right). The learned graph would be the same for all such environments (up and down actions affecting the vertical coordinate variable, and the left-right actions the horizontal one). We can store the commonly-occurring patterns like this one (another example is the velocity of an object being mirrored when colliding with an obstacle), we could represent every environment as a number of "stock" components, or recurrent independent mechanisms (coordinate variables, colliding objects) with parameters, with a small number of custom ones. Such an approach would lead to faster learning, as the agent would "guess" a component much faster, compared to learning it from scratch.
Finally, the method can be applied to challenging environments interesting in their own right. For example, the AlphaFold algorithm was applied to the protein folding problem. The model of protein folding that the agent learns is (aprori) not interpretable. However, using out method, we could obtain a sparse causal graph representing the dynamics of protein folding, which could be useful, because the problem becomes simpler. Indeed, our approach could discover interesting regularities in the way proteins fold, in terms of the final state depending only on the few variables in the initial state -- automating the process of discovery of the scientific laws of the natural world.

%%%%%%%%%%%%%%%%%%%%
\chapter{Background}
%%%%%%%%%%%%%%%%%%%%

In order to proceed, we need to introduce several important concepts: model-based reinforcement learning, causal models, methods for enforcing sparsity, methods for computing gradients for discrete variables, the Occam's razor in Solomonoff induction (supervised learning) and in the AIXI framework (Reinforcement Learning).

We use $\mathbb R$ to denote real numbers, $[n]$ to denote the first $n$ natural numbers ($[n]=\{1, 2, ..., n\}\subset \mathbb N$). For a vector with $n$ dimensions $x=(x_1,...,x_n)\in \mathbb R^n$, we use $\|x\|_p=\left(\sum_{i=1}^n|x_i|^p\right)^{1/p}$ the $p$-norm of $x$. Given a distribution $X$, we write $x\sim X$ meaning that the random variable $x$ has the distribution $X$ ($x$ "sampled from" $X$). We write $\mathbb E_{x\sim X} x$ meaning the expected value of $x$ when sampled from a distribution $X$.

\section{Reinforcement Learning}
The standard Reinforcement Learning framework consists of an {\em environment} $\mu$ and an {\em agent} $\pi$. The environment and the agent interact during an {\em episode} consisting of $T<\infty$ discrete time-steps. First, the agent receives an {\em observation} $o_1\in O$ from the environment, where $O$ is the {\em observation space}. Next, the agent executes an {\em action} $a_1\in A$ where $A$ is the {\em action space}. Next, the environment gives the agent a) a scalar reward $r_1\in R$ where $R\subseteq \mathbb R$ is the {\em reward space}, and b) the next observation $o_2\in O$. Next, either the environment terminates the interaction, in this case, $T=2$, or the cycle repeats (the agent executes an action, etc). We introduce the termination variable $d_t$ ("done") as $d_t=0$, $t<T$ and $d_T=1$.

A {\em history} (or an {\em episode}, or a {\em rollout}) of the interaction up to time-step $\tau$ consists of all the variables in the interaction in chronological order: $h_{\tau}=(o_1, a_1, r_1, o_2, ..., a_{\tau-1}, r_{\tau}, o_{\tau})$.

For each time-step, the following holds: $(o_t, r_t)\sim \mu(a_{t-1}, h_{t-1})$ and $a_t\sim \pi(h_{t})$. This means that the observation and the reward at step $t$ is defined by the environment $\mu$ given the previous history, and the action at the step $t$ is defined by the agent given the previous history.

We say that $h_T\sim (\mu, \pi)$ if the history $h_T$ was obtained from an interaction between the environment $\mu$ and the agent $\pi$.

The {\em value} is defined as the discounted sum of rewards: $V(h_T)=\sum\limits_{t=1}^T\gamma^t r_t$, where $\gamma\in[0, 1]$ is the {\em discount factor}. The goal of the agent is to find such $\pi$ that $\mathbb E_{h_T\sim (\mu, \pi)} V$ is maximized.

In our project, $A$ is discrete {\em discrete action}, such as $A=\{\mbox{up},\,\mbox{down},\,\mbox{left},\,\mbox{right}\}$\footnote{the model supports real-valued actions, they would be treated the same way as real-valued features without any difference}, and $O=\mathbb R^o$, where $o$ is the dimension of each observation. For example, $o=210\times 160\times 3$ for 210 rows, 160 columns and 3 RGB channels for an Atari observation, or $o=128$ for the $128$ RAM components in the RAM version. The reward space $R$ is bounded: $R=[r_{\min}, r_{\max}]$.

Additionally, we consider a {\em Markov} class of environments: one where next time-step only depends on the previous time-step in the history, and not on the ones before it\footnote{A typical workaround to make some non-Markov environments to comply with this property is to stack a few observations together}. For this class of environments,
$$
(o_t, r_t)\sim \mu(a_t, o_{t-1}),\,a_t\sim\pi(o_t)
$$

For example, for Cartpole, $o_t$ is obtained using the kinematics equation, and $a_t$ is computed based on the velocities and the positions of the system.

Another special important classes are {\em deterministic environments} -- ones where $\mu$ is a constant for every history (and not a distribution), and {\em deterministic agents} -- ones where $\pi$ is a constant given a history (and not a distribution).

In what follows, we consider Markov deterministic environments with deterministic agents\footnote{To account for the stochasticity, the decoder and reconstructor should include an additional sampling step, and noise variables should be additionally injected as potential causes into the model}.


\section{Model-based reinforcement learning}
Model-based reinforcement learning agent additionally includes a {\em model of the environment} $M_o$\footnote{The index $o$ means that the model works in the observation space} which is a function mapping the current observation $o_t$ and the action taken $a_t$ to the predicted next observation $\hat{o}_{t+1}$:
$$
\hat{o}_{t+1}=M_o(o_t, a_t)
$$

This function has the same signature as the environment. It is trained to fit the true dynamics: $\|o_{t+1}-\hat{o}_{t+1}\|\to\min$.

Since predicting the high-dimensional observations is computationally expensive, the observation model is often decomposed into three components: a decoder, a feature model, and a reconstructor:
$$
M_o(o_t, a_t)=R(M_f(D(o_t), a_t))
$$

Here, the decoder $D$ maps an observation into a low-dimensional latent {\em feature space} $f_t=D(o_t)$, $f_t\in \mathbb R^f$ with $f$ being the dimension of the feature space. The feature model $M_f$ predicts next features from the current ones $f_{t+1}=M_f(f_t,a_t)$, and the reconstructor $R$ maps the features back to the observation space: $\hat{o}_{t+1}=R(f_{t+1})$.

This way, the computationally expensive reconstructor can be fit on two kinds of data: end-to-end to predict the next features $\|\hat{o}_{t+1}-o_{t+1}\|\to\min$ where $\hat{o}_{t+1}=M_o(o_t, a_t)$, and to predict the current ones $\|\hat{o}_t-o_t\|\to\min$, where $\hat{o}_t=R(D(o_t))\equiv R(f_t)$. This way, the convergence is sped up.

Another technique prevents predicting high-dimensional observations altogether by replacing $R$ with a value function predictor, which is fit to predict the reward-to-go $R_t(h_T)=\sum\limits_{\tau=t}^{T}\gamma^{\tau-t+1}r_{\tau}$:
$\|\hat{R}_t-R_t\|\to\min$ where $\hat{R}_t=R_{value}(f_t)$.

\section{Causal modeling and deep causal modeling}

\section{Enforcing sparsity in deep architectures}

\section{Methods to compute gradients of discrete variables}

\section{The AIXI framework and its connection to our work}
\subsection{Solomonoff induction}
\subsection{The AIXI framework}

The background section introduces the necessary background to understand your
work. This is not necessarily related work but technologies and dependencies
that must be resolved to understand your design and implementation.

This section is usually 3-5 pages.


%%%%%%%%%%%%%%%%
\chapter{Design}
%%%%%%%%%%%%%%%%

Introduce and discuss the design decisions that you made during this project.
Highlight why individual decisions are important and/or necessary. Discuss
how the design fits together.

This section is usually 5-10 pages.


%%%%%%%%%%%%%%%%%%%%%%%%
\chapter{Implementation}
%%%%%%%%%%%%%%%%%%%%%%%%

The implementation covers some of the implementation details of your project.
This is not intended to be a low level description of every line of code that
you wrote but covers the implementation aspects of the projects.

This section is usually 3-5 pages.


%%%%%%%%%%%%%%%%%%%%
\chapter{Evaluation}
%%%%%%%%%%%%%%%%%%%%

In the evaluation you convince the reader that your design works as intended.
Describe the evaluation setup, the designed experiments, and how the
experiments showcase the individual points you want to prove.

This section is usually 5-10 pages.


%%%%%%%%%%%%%%%%%%%%%%
\chapter{Related Work}
%%%%%%%%%%%%%%%%%%%%%%

The related work section covers closely related work. Here you can highlight
the related work, how it solved the problem, and why it solved a different
problem. Do not play down the importance of related work, all of these
systems have been published and evaluated! Say what is different and how
you overcome some of the weaknesses of related work by discussing the
trade-offs. Stay positive!

This section is usually 3-5 pages.


%%%%%%%%%%%%%%%%%%%%
\chapter{Conclusion}
%%%%%%%%%%%%%%%%%%%%

In the conclusion you repeat the main result and finalize the discussion of
your project. Mention the core results and why as well as how your system
advances the status quo.

\cleardoublepage
\phantomsection
\addcontentsline{toc}{chapter}{Bibliography}
\printbibliography

% Appendices are optional
% \appendix
% %%%%%%%%%%%%%%%%%%%%%%%%%%%%%%%%%%%%%%
% \chapter{How to make a transmogrifier}
% %%%%%%%%%%%%%%%%%%%%%%%%%%%%%%%%%%%%%%
%
% In case you ever need an (optional) appendix.
%
% You need the following items:
% \begin{itemize}
% \item A box
% \item Crayons
% \item A self-aware 5-year old
% \end{itemize}

\section{Introduction}
Consciousness prior, AIXI lead to simple or sparse models.

Disentangled representations and representation learning

Model-based reinforcement learning

Causal modelling for reinforcement learning. Interventions

AI safety: distributional shift robustness (generalization), interpretability, sample-efficiency

\section{Proposed architecture}
\subsection{Theory}
Like in AIXI, would like to find the simplest model of the environment:
$K(\mu)\to\min$ s.t. $L(\mu, data)\to\min$. In general, this problem is uncomputable.

We decompose the model into a Decoder and a sparse Model. The model in observational space $W$ is then the sparse Model $M$ applied to Decoded features, with the result projected back into the high-dimensional space.
$$
W=D^{-1}MD
$$

\subsubsection{Why sparse models?}
Traditionally, sparsity of models usually means making the number of non-zero parameters low. This can be seen in linear regression when applying the Lasso regularization. Such methods learn to rely only on a handful of features, instead of considering the whole input. When data is multicollinear, this leads to better generalization, as the model only relies on the most salient features, and is less likely to learn spurious correlations between features.

In Supervised Deep Learning, sparsity of representation traditionally means regularizing the activation of the latent hidden layer in an autoencoder-like setup (an encoder transforming a high-dimensional representation into a low-dimensional one, and then a decoder doing the inverse transform back to the high-dimensional space). In such setups, sparsity is enforced in order to enforce constraints on the learned latent space. With such regularization, the model is learning a sparse coding -- one where each input is represented with a minimal number of non-zero components.
This representation is better than a "dense" one, in case if there are certain regularities in the dataset.
For example, if the input images have an underlying latent generative model in which the high-dimensional image is a "child" of only few discrete variables (such as, a cat can only be of certain colors, which can be represented in a one-hot way, and other images, such as images of dogs, can be distinguished from images of cats by another binary variable).
In neuroscience, such representations are biologically plausible due to lower energy usage and higher fault tolerance [??].

However, sparse representation does not necessarily imply interpretability. In supervised tasks, it it known that [Google post on disentanglement] the model still learns spurious correlations, which leads to different controllable aspects of the image being "entangled" in one activation node.

Spurious correlations in supervised learning do not have a general way of being resolved, because observational data is not enough to learn the correct causal model. An example of this is adversarial images, which can be seen as spurious correlations, namely, the model relying on high-frequency features which, while discriminating between images on the training set due to high memorizing capacity of the network, do not generalize to the shifted adversarial distribution [features not bugs post].

While the problem of learning good models from observational data is interesting in its own right, in this project we focus on a reinforcement learning setting where there is a possibility to perform actions. Since switching from one policy to another determines the distribution of histories, switching from one policy to another can be seen as performing interventions on the underlying causal graph. In this way, for example, an agent playing Breakout by precisely targetting each block one-by-one is different from an agent trying to put the ball to the top of the playing field, speeding up the process significantly. If we consider the underlying data generation process in the latent space of positions of various objects (x and y coordinates of the ball, the player and all the targets), and in an interaction between them, and a similar representation for the agent (for example, the first agent's action depends on the leftmost still-standing target, which it hits perfectly, and the second agent's action depending as well on whether or not there is an opportunity to get to the top), switching from one policy to another can be seen as an intervention in this history-generating graphical model, in which the gate switching the dependencies of the policy's action is altered, and set to $1$ instead of $0$, signifying the reliance to the second strategy.

While the total space of policies is represented with exponentially many such switch nodes (namely, one node for each subset of coordinates the policy can possible depend on, and one node for every discretized value of the weights of a neural network, for example), some of them are more complex than others.

We start the theoretical derivation from the AIXI framework, where the environment is a Turing machine, and the agent is a non-computable greedy best-response Bayesian solution considering all possible environments. Theoretically, such agent is known to have certain optimality conditions under the assumption of sufficient exploration. Compared to Solomonoff  induction (a perfect uncomputable Bayesian solution to the online supervised learning problem, or the task that the GPT-3 solves -- self-supervised prediction), in RL, AIXI (which is designed in the same way as the Solomonoff induction) is not always optimal. Indeed, it was discovered that AIXI does not explore enough, because there are bad priors leading to the agent "believing" that it is not beneficial to explore. In the simplest setting, we can always select such a prior where the probability of the true environment, and of any sufficiently close-by environment leading to rewards when taking action $a_1$ is arbitrarily low, leading to the agent never trying $a_1$, and, thus, never updating its posterior [On the optimality...].

General theory of RL suggests solutions to this problem, leading to weaker convergence results [Thompson sampling]. The basic idea is to let go of the agent being perfectly Bayesian, and (knowingly) use an outdated model to take actions. Intuitively, this 'denial' leads to even improbable actions being taken into consideration, and, thus, has more chances of discovering the true posterior.

The Solomonoff Bayesian setting considers weighting the environment models by their complexity.

In this setting, we consider the prior over environments $\mu$, $\xi(\mu)=2^{-K(\mu)}$ where $K$ is the Kolmogorov complexity. Next, we collect history using a policy $\pi$, which gives a Bayesian posterior:

$$
\nu(\mu|h)=\frac{\mu(h)2^{-K(\mu)}}{\sum_{\nu}\nu(h)2^{-K(\nu)}}=\frac{\mu(h)\xi(\mu)}{\xi(h)}
$$

Given the posterior $\nu'=\nu|h\equiv \nu(\mu|h)$, we "simply" run an infinite tree search to find the best possible action:
$$
a_t^{AIXI}=\arg\max_{a_t}\sum\limits_{t=t_0}^{\infty}\gamma^t\sum_{r_t,s_t}\nu'(r_t,s_t|r_{<t}s_{<t}a_{<t})
$$

Here, $\xi(h)$ is the prior probability of the data, or the probability of history in the Universal Environment (a mixture of environments where each environment is weighted with its complexity).

In practice, it is not viable to search over all possible Turing machines (which is required to evaluate the denominator, $\xi(h)$), and we cannot run them for an infinite number of time-steps (like it is done in AIXI). Another problem here is that we cannot compute $K(\mu)$.

To make the problem practical, we additionally assume that there is a structure to the environment.

Namely, we assume that for the whole environment there exist a mapping $f\colon o\to f$ which maps observations to features. For a while, we assume deterministic environments. We also assume 1 time-step dependencies (Markov property). Given our feature assumption, we assume that in the feature space, the dynamics of the environment is linear (inherent simplicity??). We only consider environments $\mu$ which are representable (we consider $r_t$ as part of $s_t$ for simplicity of notation here) as
$$
f_t(\mu)=M_ff_t+M_aa_t+b
$$

{\bf OMG}

Our linear class $f_{t+1}=M_ff_t+M_aa_t+b$ is a really small class of environments, even with a non-linear decoder... Indeed, a Line environment (navigating on a line with 3 states) does not fit into that definition. Consider an MDP with 3 states, $[1, 2, 3]$, and two actions, right and left encoded as one-hot: $a_1=[1, 0]$ and $a_2=[0, 1]$. The bias term in the linear model can be shifted into the actions $M_a$. Suppose that the decoder gives some features for the states $f_1, f_2, f_3$ (does not matter how, it's only important that the features are fixed). Multiplying $M_aa_i=A_i$ by definition, this gives two vectors $A_1$ and $A_2$.

Then, $f_1=M_ff_1+A_2$ (going left from $f_1$ gives $f_1$), and $f_3=M_ff_3+A_1$ (going right from $f_3$ gives $f_3$). Now consider what happens in $f_2$. If we go left, we get $f_1$: $M_ff_2+A_2=f_1$. If we go right, we get $f_3$: $M_ff_2+A_1=f_3$. Subtracting these two gives $f_3-f_1=A_1-A_2$. On the other hand, the "stuck at the wall" equations give us $f_3-f_1=M_ff_3+A_1-M_ff_1-A_2=M_f(f_3-f_1)+A_1-A_2$. So, equating these two, we get
$f_3-f_1=M_f(f_3-f_1)+A_1-A_2=A_1-A_2$. This means that $M_ff_3=M_ff_1$.

Now, consider the "stuck at the right wall" $f_3=M_ff_3+A_1$, and going right from $f_1$: $f_2=M_ff_1+A_1$. But $M_ff_1=M_ff_3$, which means that $f_2=M_ff_3+A_1$. Note the same RHS as for the "stuck at the right wall". Therefore, $f_2=f_3$, {\bf we cannot distinguish between $f_2$ and $f_3$, degenerate representation.}



Here, $f_t=D(s_t)$ and $s_t=R(f_t)$

Imagine if the representation $f$ is sparse at every time-step. Does this guarantee an interpretable environment?

From the point of view of the Free Energy principle, we are looking for a minimal Markov Blanket.

From the point of view of the leading theory of consciousness, Integrated Information Theory. Why do we look for sparsity, when consciousness there is about irreducibility of a big model into smaller models? IIT does not specify which representation of the physical world needs to be mapped to the nodes in the causal graph, while we have a decoder which learns this representation, and selects one where the model is the simplest one. Tononi says that we should pick the representation with maximal $\phi$, but, it seems, In our case, something not affecting the reward (or the reconstruction in general), would be independent of causal variables that are optimized for. Thus, such a variable would be unconscious. The project can be seen as computing a proxy for $\phi$ -- minimizing the complexity. Such an interpretation of IIT (allowing for an arbitrary function computing the representation, and minimizing over complexities/$\phi$) could resolve the issues raised in \cite{doerig2019unfolding}. There, a function is applied to the representation, after which an inverse transform is applied. If we only care about the minimal-$\phi$ representation, all of that would be cancelled, and only the high-level variables would remain.


What if the decoder is too complex? We could re-use components, like humans don't need to re-learn existing conv1 filters to learn a new game.

In the model, we can sample from a categorical gumbel-softmax distribution in order to select a model. We can re-use models for different features. Each of the feature models has a categorical distribution over the bank of models, and we regularize for the total entropy of all models (with an addition that we can permute the features in arbitrary way before applying the model).


The total complexity of the environment $\mu$ is thus $K(\mu)=K(M_a)+K(M_f)+K(D)+K(R)$


$\mu(h)\equiv \mu()$

Finding good features by regularizing the {\em model} to be sparse.

Three losses: model fit, sparsity regularization, reconstruction

Reconstruction losses: inverse decoder norm, reconstructing observations (autoencoder), inverse model norm (we force the model to be an invertible matrix), predicting actions, predicting value function from features (like in muzero)

Causal learners: linear, graph NNs, linear with sampling

For the model fit, we use Mean Squared Error. Alternatives are to use contrastive loss (with fake samples)

For the sparsity, we use L1-regularization. An alternative is to use projection.

Architecture improvements: model at each layer of the decoder for a faster fit, many time-steps, batch normalization, adaptive sparsity loss

\section{Literature review}

Simple mathematical equations can describe our world well \cite{hamming1980unreasonable}. We assume that the environments in RL that we are interested in are also describable by such equations.

\cite{Battaglia2018} introduces Graph Neural Networks (GNNs) based on the idea of thoughts as graphs of connections between abstract concepts. The idea is to combine the strengths of "handcoded" or "biased" models and the "end-to-end" or "from scratch" models. \cite{Velickovic2020} extend this idea by allowing a linear (in the number of nodes) number the edges to be learned.

\cite{Cranmer2020} use GNNs to discover unknown laws of dynamics. They apply symbolic regression to the node and edge models of the GNN, to extract the symbolic expression. The model is used for pairwise forces and for discovering the laws for Dark Matter. The symbolic expression (Eureqa) generalizes better than the non-symbolic model it was extracted from. $l_1$ sparsity is used in the GNN to ease the problem for the symbolic regression.

Libraries: pytorch-geometric, graph-nets from DeepMind

\cite{Genewein} propose to use probability trees instead of graphical models to encode causal relationships. The benefit is that the set of parents of a node can be dynamically dependent on values of other nodes. The paper does not provide a way to construct the probability trees.

\cite{Zambaldi2018} introduce an attention module inside a Reinforcement Learning agent (inside the policy and the value networks), allowing it to solve tasks not solved by a simple MLP, because the task requires relational reasoning. The attention on the charts seem to favor one object attended to, corresponding to the ground truth next item to go to. Similar technique can be seen in \cite{Hahne2019} to solve visual reasoning tasks (similar to IQ tests)

\cite{Xie2020} introduce latent variables into the learned linear causal models using statistical tests.

\cite{Johnson2016} introduce a variational approach to model dynamics: probabilistic graphical model in terms of Switching Linear dynamics is applied to a learned embedding using a VAE. The model is applied to 1D and 2D image tasks, and it successfully uncovers the underlying dynamics.

\cite{VanDenOord2018} fit a neural autoregressive model. The paper suggests that fitting many-step future predictions is of utter importance, since only in such predictions the model is tested for "true" generalization of high-level features. The paper also suggests that MSE or other losses commonly used for reconstruction are not optimal, and generative models are hard to train. As an alternative, the paper suggests optimizing for the mutual information between the latent code and the original inputs. The model predicts the log-odds ratio for an offset $k$ and input up to time $t$ as $f_k(x_{t+k},c_t)\sim \frac{p(x_{t+k}|c_t)}{p(x_{t+k})}$ with a model $f_k(x_{t+k}, c_t)=e^{D(x_{t+k})^TW_kc_t}$. This model is trained using the InfoNCE loss (noise-contrastive estimation) $L=-\mathbb E_X\log\frac{f_k(x_{t+k}, c_t)}{\sum_{x_j\in X}f_k(x_j,c_t)}$. The dataset $X$ consists of one sample $x\sim p(x_{t+k}|c_t)$ (ground truth) and negative samples $x\sim p(x_{t+k})$ (the prior).

When using probabilistic inference with gradient descent-like algorithms, we need to differentiate through a sampled value $X$ with respect to the sampled parameters $X\sim X(\theta)$. For discrete distributions, reparametrization trick would not work directly, as the gradient of argmax is zero a.s. One version is the log-likelihood trick, but it results in high variance. Another approach is to replace the hard discrete distribution with a soft one, one version of which is the Concrete distribution \cite{Maddison2017}.

\cite{kalainathan2018structural, ng2019masked} propose to learn a distribution over causal graph in a mean-field way (all edges are independent and modelled as Bernoulli distribution). \cite{Brouillard2020} extend this approach to modelling interventions.

\cite{Fallah2020} use dictionary learning with an autoencoder.

\cite{kipf2019contrastive} work on learning abstract representations from pixels in 3D environments and in Atari using Graph neural networks, leading to more interpretable world (environment) models.

\section{Model learning for a fixed policy}
The problem of learning a model is formulated as follows. Given a deterministic RL environment $\mu$ and a policy $\pi$, we want to find a function computing the next observation: $o_{t+1}=M_o(o_t, a_t)$. Additionally, we would like to learn the reward and the termination ("done"): $r_t=M_r(o_t, a_t)$, $d_t=M_d(o_t, a_t)$. Since observations are high-dimensional, we require the model to be represented as a composition of a decoder, a model in the feature space, and a reconstructor: $o_{t+1}=R(M_f(D(o_t), a_t))$. If we denote $f_t=D(o_t)$, we can additionally require $D(o_{t+1})\approx M_f(f_t, a_t)$. In this case, the function $M_f$ is a {\em model in feature space}. If we additionally require $R(D(o_t))\approx o_t$, the function $R$ becomes the inverse for the decoder.

Training the reconstructor is usually the most time-consuming part, since it requires to train a generative model for high-dimensional observations. Therefore, several techniques are applicable to avoid this part. In these cases, a full model in observation space is not obtained.
\begin{enumerate}
    \item Margin loss in the space of observations. For observations $o_1\neq o_2$, we require that $\rho(D(o_1), D(o_2))\geq h$ via a margin loss: $\max(h-\rho, 0)\to\min$
    \item Contrastive loss: we train a classifier $C(f, f')$ with a loss $\frac{C(f_t, f_{t+1})}{C(f_t, f_{t+1})+C(f_t, f_{neg})}\to\max$ where $f_{neg}$ is not equal to the true feature $f_{t+1}$
    \item Training a discriminator with labels $C(f_t, f_{t+1})\approx 1$, $C(f_t, M_f(f_t, a_t))\approx 0$, and with a separate optimizer training for $C(f_t, M_f(f_t, a_t))\approx 1$ (GAN)
    \item Instead of predicting high-dimensional observations, only predicting some sufficient (in the sense of playing the game) statistic, such as the value function: $V_f(D(o_t))\approx V(o_t)$, where $V_f$ is a trainable value-to-go predictor, and $V(o_t)$ is the reward-to-go collected at the episode.
\end{enumerate}

{\bf Sampling procedure.} Since we would like the model to approximate the next observation $o_{t+1}\approx M_o(o_t, a_t)$, we must sample triples $(o_{t+1}, o_t, a_t)$ (and other related features such as reward, termination, or reward-to-go) from the uniform distribution: we randomly sample an episode, and then pick a random observation. This would mean that we only keep one step from a whole episode. This is not practical, as we have to "throw away" most of the steps. We approximate this with an experience replay buffer. In case if the environment "looks differently" for different initialization random seeds (for example, the labyrinth has a different layout), such an approximation results in the following issue. If we sample a mini-batch from 1 episode, and the subsequent mini-batch from another (single) episode, the model would tend to overfit to the first mini-batch (specifically, for example, to the locations of objects in the labyrinth). On the next call, the loss will spike, and the model will overfit to the next object locations. Having an experience replay buffer and sampling the triples randomly from it alleviates the issue. This is the same problem as was encountered in the DQN paper.

In practice, we use the mean-squared regression. To account for different scales of observation components, feature components and rewards, we compute the {\em relative} mean absolute error:
$$
L_{rel-mse}(f^{pred}, f^{true})=\frac{1}{T}\sum\limits_{t=1}^T\sum\limits_{f=1}^F\left[\frac{f^{true}_{f,t}-f^{pred}_{f, t}}{\sigma'(f^{true}_{f, \cdot})}\right]^2
$$

Here $\sigma(f^{true}_{f, \cdot})$ is the standard deviation for the true features. In case if this value is less than $\varepsilon=10^{-8}$, we set this value to $1$. This is required for the case of constant pixels, or constant features. Setting this value to $\varepsilon$ instead of $1$ produces extremely high loss spikes in case if in some of the batches, a feature is constant, and in others, it is not.

The relative loss $L_{rel-mse}$ is more interpretable than the standard mean-squared-error loss, because its value signifies how many features have a large error. For example, for $10$ equal features with equal predictions, a value $L_{rel-mse}=10$ means that for all features, the prediction error is equal to the standard deviation. More importantly, a value $L_{rel-mse}=10^{-2}$ means that for each of the features, the relative error is at most $10\%$ in terms of the standard deviation.

\begin{itemize}
    \item Reconstruction loss: $L_{rec}=L_{rel-mse}(R(D(o_{x})), o_{x})$. Here we denote values before (and while) taking an action as $x$, and values at a step immediately after taking an action as $y$. In this notation, for example, $o_{x, t+1}=o_{y,t}$
    \item Prediction loss in observation space: $L_{pred}=L_{rel-mse}(R(M_f(D(o_{x}, a_{x})), o_{y})$
    \item Feature consistency loss in feature space: $L_{cons}=L_{rel-mse}(M_f(D(o_x, a_x)), D(o_y))$
\end{itemize}

Both of the losses $L_{pred}$ and $L_{cons}$ are required to obtain good features. First, training with $L_{pred}$ allows to have an end-to-end model, where the right-hand side does not depend on the decoder. This becomes a supervised learning problem. Secondly, without optimizing for the feature consistency loss, $L_{pred}=0$ does not guarantee that $L_{cons}=0$, since the reconstructor can "recognize" two different sets of features. In this case, output features from the model do not necessarily correspond to the input features.

Since we would like to obtain an interpretable model, we add the feature consistency loss $L_{cons}$.

{\bf Separate models for each feature and each pixel.} Empirically, in our test toy environments (VectorIncrement and KeyChest), the tasks of predicting different pixels with a reconstructor $R$ do not have many information in common, and, therefore, are poor tasks for multi-task learning. Empirically, predicting each pixel $i$ with a separate model $R_i(f)$ converges faster than having a joint neural network $R(f)$.

\section{Sparse model learning}
Suppose that we have reasonably low losses $L_{rec},\,L_{pred},\,L_{cons}<\varepsilon$. This means that for each of the output pixels, we can predict it with relative error at most $\varepsilon$. In addition, the same applies to the relative error in predicting each feature.

In the space of all reconstructors, decoders and models, we would like to find a model which is most sparse in the sense of a causal graph. This means that, for each output feature $f_j$, we would like it to depend on the fewest amount of input features $f_i$. Given that, the graph of dependencies in the feature space is most sparse (in terms of the number of edges), or, the program describing the environment is most simple.

To do so, we use the model from Bengio. We use a tensor of probabilities of shape $p=(F+F', F+A)$, where $F$ is the number of features, $F'$ is the number of additional features to predict (2 -- reward and termination), and $A$ is the number of actions. This tensor has values between $0$ and $1$. The values are interpreted as probabilities of having an edge from feature $f_i$ to feature $f_j$: if the value is $1$, the feature $f_j$ depends on $f_i$ with probability $1$.

Given such a tensor of probabilities, we can sample random variables $\xi\sim p$, where $\xi$ is a binary matrix of dimensions $(F+F', F+A)$ with values in $\{0, 1\}$. Now, if an edge is not present in the graph, the model for a feature should not depend on the corresponding input feature. We achieve this by multiplying the input with this tensor $\xi$. Next, we have $F+F'$ models predicting each feature, each depending on the input multiplied with the binary mask $\xi$.

Having $F+F'$ independent models instead of having one big model with $F+F'$ outputs has the same advantage as stated before for the reconstructor $R$: since the tasks do not have a lot in common, the convergence is faster if the networks are independent.

In addition, the model receives the mask $\xi$ apart from the input $[f_t, a_t]\odot \xi_i$. This is necessary because the model needs to "know" whether it has received $0$ as input because the feature is $0$, or because the feature was "de-selected".

Suppose that we have a dataset of triples $\{(f_{t+1}, f_t, a_t)\}$. We sample masks $\xi_t$ for every time-step $t$. For this example, we set $p_{ji}=0.5$ for all features. The loss $L_{pred}$ and $L_{cons}$ becomes a random variable, because the inputs a pre-multiplied by random masks. If we fit the model on this data (in a supervised way), it will learn to rely on features if they are present, and to replace them with a "mean value" in case if they are not. We can optimize the parameters of $R, M, D$ using gradient descent. Now, how to obtain the gradient with respect to $p$?

The well-known trick is Gumbel-Softmax, which can yield $\frac{\partial L}{\partial p}$ given samples. However, since it internally uses saturating softmax,  it takes a lot of time in practice to obtain a value close to $1$. Even in the mostly-linear regime, it takes a lot of time to optimize for parameters $p$, because the gradient is multiplied by $\sigma(\cdot)(1-\sigma(\cdot))$. Instead, we use the REINFORCE gradient:

$\frac{\partial L}{\partial p}=\frac{\partial}{\partial p}\mathbb{E} L(\xi(p))=\frac{\partial}{\partial p}\mathbb E_{\xi_{11}}...\mathbb E_{\xi_{nm}}L(\xi_{11},...,\xi_{nm})$.

Now, consider one-dimensional case and $\frac{d}{dp}\mathbb E_{\xi\sim F(p)}L(\xi)=\int \frac{d}{dp}f_p(\xi)L(\xi)d\xi$. Now, using the log-likelihood trick, $\frac{df_p(\xi)}{dp}=\frac{d\log f_p(\xi)}{dp}f_p(\xi)$. Therefore, $\frac{d}{dp}\mathbb E_{\xi\sim F(p)}L(\xi)=\int\frac{d\log f_p(\xi)}{dp}f_p(\xi)L(\xi)d\xi=\mathbb E_{\xi\sim F(p)}\frac{d\log f_p(\xi)}{dp}L(\xi)$. In case of discrete distributions, we need to replace the probability density function $f_p(\xi)$ with the probability mass function $F_p(\xi)$

For the Bernoulli distribution, this gives $\frac{d}{dp}\mathbb E_{\xi\sim Be(p)}=\sum_{\xi\in\{0, 1\}}\frac{d\log F_p(\xi)}{dp}L(\xi)$. Since $F_p(1)=p$ and $F_p(0)=1-p$, we obtain $\frac{d}{dp}\mathbb E_{\xi\sim Be(p)}=L(\xi=1)-L(\xi=0)$.

In case if we have many random variables $\xi_{ji}$, we can set $L(\xi_{ji})=\mathbb E_{\xi_{-ji}}L(\xi_{ji}, \xi_{-ji})$, and then obtain $\frac{\partial }{\partial p}L_{rel-mse}=\mathbb E_{\xi_{-ji}}L_{rel-mse}(\xi_{ji}=1, \xi_{-ji})-\mathbb E_{\xi_{-ji}}L_{rel-mse}(\xi_{ji}=0, \xi_{-ji})$.

In practice, given a sample of $\xi$ of shape $[T, F+F', F+A]$, for a pair of features $j, i$, we select the values of the loss $L_{t}$ where $\xi_{t, j, i}=1$, and take the average of such $L_t$ to obtain a sample estimate of $\mathbb E_{\xi_{-ji}}L_{rel-mse}(\xi_{ji}, \xi_{-ji})$. The same procedure is applied for the values of $t$ where $\xi_{t, j, i}=0$. This corresponds to running $F+F'\times F\times A$ "randomized control trials" to establish the effect of $\xi_{ji}$ on $L$, and computing the difference in means. This way, we manually compute the gradient for each $p_{ji}$.

While REINFORCE gradients are known to be noisy, here we only have two values, and sample sizes are considerably large for large batch sizes $T\sim 5000$ and $p\sim 0.5$. By the central limit theorem, the sample means are close to the true expectations. After applying the gradient, the probabilities $p$ are clipped to be between $0$ and $1$.

Note that the problems of estimating the gradient are not independent, in a sense that a gradient with respect to $\xi_{ji}$ depends on values of $p_{-ji}$. This happens because the values of the loss depend on these features as well.

Let's consider different cases, assuming that the model was fitted to convergence on a set of features with a fixed tensor of probabilities. Consider the case of $2$ input and output features, with features being equal and having a deterministic relationship (from input to output).
\begin{enumerate}
    \item $p=0$ for all pairs. In this case, no features are given to the model, and the model will learn to predict the mean of the output features. The causal graph is empty. The loss is equal to $2$ (since standard deviation for each feature is $1$ w.l.o.g. because of the relative error).
    \item $p=1$ for all pairs. In this case, the model uses all features, and the error is minimal. The loss is $0$.
    \item $p=0.5$ for all pairs. In this case, the model will try to extract as much information as possible when the feature is present. Essentially, with probability $0.25$, the model will give the mean output feature, and with probability $0.75$ it will use either one of the input features, or both of them. The loss is equal to $2\times 0.25=0.5$.
    \item In the general case $p\in [0, 1]$, the model has an error of $2$ with probability $(1-p)^2$ and a loss of $0$ with probability $1-(1-p)^2$. The expectation is $2(1-p)^2$.
\end{enumerate}

Now consider a similar case when the mutual information between features is $0$ (they are independent). In this case, for a probability $p$, we have the loss being $2$ with probability $(1-p)^2$, a loss of $1$ with probability $2p(1-p)$ and a loss of $0$ with probability $p^2$. The expectation is $2(1-p)^2+2p(1-p)=2-4p+2p=2-2p=2(1-p)$. This loss is higher than for the case of equal features for all $p\in (0,1)$.

To avoid a case when there are no samples with $\xi=1$ (with low $p$), we limit the range of $p$ to be $p\geq p_{\min}=0.01=1\%$. Setting this value higher results in feature duplicates, as the model tries to predict the outputs even from partially present features. A value of $1\%$ with a batch size of $T=5000$ gives $50$ samples with $\xi=1$ in the worse case (on average), which is empirically sufficient to turn features on. To turn the features off, we introduce a sparsity regularizer. To ease this even more, we add a version of the losses with $p=0.5$ (regardless of the true value of $p$). To train the model faster, we add a version of the loss with $p=1$ as well.

Denote $L_{model}=L_{pred}+L_{cons}$. Given the two other versions with $p=0.5$ and $p=1$ we have also $L_{model}^{p=0.5}$ and $L_{model}^{p=1}$.

\subsection{Annealing.} To learn a sparse model, we additionally regularize the probabilities $p$ for sparsity as $L_{sparse}=\sum_{ij}|p_{ij}|$. The total gradient for $p$ is a sum of gradients from $L_{sparse}$ and the REINFORCE gradient from $L_{pred}$ and $L_{cons}$.

Consider the relative error ("relative sparsity gap") $\gamma(R, D, M, p)=\frac{L_{model}^{p=p}-L_{model}^{p=1}}{L_{model}^{p=p}}$. It shows how the loss given the current selected features compares to enabling all features. For a any fixed $p$ and any $D$, if we fit the $M$ and $R$ on $L_{model}$ and $L_{model}^{p=1}$ (note that there is no conflict between these two losses, because masks uniquely determine whether the features are on or off. Essentially, this is a big supervised dataset), the value of $\gamma\in[0, 1]$: turning off some features can only increase the loss. In case if $p<1$ and $\gamma=0$, the output does not depend on the input. If $\gamma=1$, the model perfectly fits the data ($L=0)$ with $p=1$ (this does not happen in practice). $\gamma=0.5$ means that the error could decrease $50\%$ in case if we turn on all features.

An additive error $\delta(R, D, M, p)=L_{model}^{p=p}-L_{model}^{p=1}$ is in range between $0$ (if model is trained) and $N=F+F'$ (in case if model can perfectly fit the data and $p=0$). Therefore, we consider the ratio $\Delta(R, D, M, p)=\frac{\delta(R, D, M, p)}{F+F'}\in[0, 1]$.

Annealing proceeds in the following stages:
\begin{enumerate}
    \item $\delta>\delta_{\max}$. In this case, the model is trained to decrease $L_{model}^p$, which increases some $p$. Sparsity coefficient is decreased (we care more about fitting any model rather than about sparsity).
    \item $\delta\leq \delta_{\max}$. In this case, we care about sparsity and increase the coefficient. In case if at some point, $\delta$ crosses the threshold, we do not change the coefficient for some time, to allow the model to learn from a dataset where some features are turned off. This "freezing" at a "low temperature" allows the model to "crystallize" and learn a new representation (one where some features are turned off).
\end{enumerate}

Consider the following cases:
\begin{enumerate}
    \item $p=p_{\min}$. In this case, the model cannot predict features reliably: even in the toy example above, the error would be $L\sim 2\times (1-p)\approx 2$ for 2 features -- very bad result for a relative error. In this case, the REINFORCE gradient will turn on features yielding maximal decrease in the loss.
    \item $p=1$. In this case, we would like to turn off some of the features, because we would like to obtain the most sparse model.
\end{enumerate}

Overall we would like to find the sparsest graph such that the sparsity gap does not exceed some $\varepsilon_{\max}$:
$$
\min L_{sparse}\,s.t.\,\delta\leq \varepsilon_{\max},L_{model}^{p=1}+L_{rec}\leq \varepsilon_{loss}
$$

This is a constrained optimization problem. There are several approaches to this problem:
\begin{enumerate}
    \item Static linear combination: $\frac{L_{sparse}}{NM}+\frac{L_{model}^{p=p}+L_{model}^{p=1}}{2N}+\frac{L_{rec}}{O}$
    \item Dynamic linear combination: $\frac{L_{rec}}{O}+\frac{L_{model}^{p=1}+L_{model}^{p=p}}{2N}+\beta L_{sparse}$ where $\beta$ is chosen based on the value of $\delta$ as described above
    \item Lagrange multipliers. We consider $\mathcal L(\lambda, D, M, R)=L_{sparse}+\lambda (L_{rest}-\varepsilon_{rest})$. The point with $\max_{\lambda}\min_{D, M, R}\mathcal L$ corresponds to $L_{sparse}\to\min$ s.t. $L_{rest}\leq\varepsilon_{rest}$. The standard solution is a primal-dual method:
    $$\lambda_{t+1}=\lambda+\eta\frac{\partial \mathcal L}{\partial \lambda}$$
    $$(D, M,R)_{t+1}=(D,M,R)-\eta\frac{\partial\mathcal L}{\partial (D, M, R)}$$
\end{enumerate}

\section{Environments}
\begin{enumerate}
    \item VectorIncrement. Recovering a sparse linear matrix from skewed observations
    \item KeyChest. Learning a sparse causal graph in a grid-world
    \item Physics environments. Learning physics laws
    \item Atari: MsPacman, Montezuma Revenge. Learning a better representation than pixels
    \item PyGame learning environment (flapping bird)
    \item Starcraft. Learning in high-dimensional environments
    \item VIZDoom. Learning abstract features
    \item TextWorld. Extracting a low-dimensional model governing the decisions
    \item OpenSpiel. Finding good representations automatically for known games
\end{enumerate}

\section{Results}

\begin{enumerate}
    \item Can fit a sparse model on VectorIncrement with pinverse sparsity loss, fully-connected nets and value function for reconstruction
    \item Can train agent on KeyChest (DQN and PPO) demonstrating that it is Markov
    \item Can fit a sparse model (without sparsity regularization) ignoring actions that somewhat well predicts the health (but not health increase, only decrease)
    \item Cannot fit a VAE, AE, VAE-GAN on KeyChest data for some reason
\end{enumerate}

\subsection{What does MuZero learn}
\subsection{Learned graphs for our environments}
\subsection{Game complexities}
\subsection{Properties of the new agent}
\subsubsection{Spurious correlations and exploration speed}
\subsubsection{Sample efficiency}
\subsubsection{Online DS robustness (re-training speed)}
\subsubsection{Offline DS robustness (deploying into the environment)}


\end{document}
